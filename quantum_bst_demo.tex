\documentclass[a4paper,twocolumn,11pt]{quantumarticle}
\pdfoutput=1
\usepackage[utf8]{inputenc}
\usepackage[english]{babel}
\usepackage[T1]{fontenc}
\usepackage{amsmath}
\usepackage{amssymb}
\usepackage{hyperref}
\usepackage{csquotes}
\usepackage[resetlabels]{multibib}
\newcites{article}{Article references}
\newcites{repo}{Repository references}
\newcites{web}{Website references}
\newcites{book}{Book references}
\newcites{tests}{Test references}

\begin{document}
	\title{Template demonstrating the quantum bibstyle}
	
	\author{David~Wierichs}
	\affiliation{Institute for Theoretical Physics, University of Cologne, Germany}
    \author{Johannes~Jakob~Meyer}
    \affiliation{Dahlem Center for Complex Quantum Systems, Freie Universit\"{a}t Berlin, 14195 Berlin, Germany}
    \affiliation{QMATH, Department of Mathematical Sciences, University of Copenhagen, 2100 Copenhagen, Denmark}

	\maketitle
	\onecolumn
	\section{Reference class \texttt{article}}
	For the \texttt{article} class, the \texttt{title} is printed in \emph{italics}. The \texttt{journal} is not reformatted, the \texttt{volume} printed in \textbf{bold font}. We also include the \texttt{pages} if present and the \texttt{year} in round brackets ().
	\texttt{doi} links are always included if given, the same holds for \texttt{eprint}. Only if neither of these two fields is given do we use the \texttt{url} to provide a hyperlink to the article.
	Code repositories are linked whenever provided via the \texttt{code} field, which is a non-standard field in 
	quantum.bst. 
	
	Examples:
	
	\begin{tabular}{ccccc}
		\texttt{doi}& \texttt{eprint} & \texttt{url} & \texttt{code} & result \\
		$\checkmark$ & $\checkmark$ & $\checkmark\big / \times$ & $\checkmark$ & \citearticle{article_doi_eprint_url_code} \\
		$\checkmark$ & $\checkmark$ & $\checkmark\big / \times$ & $\times$ & \citearticle{article_doi_eprint_url} \\
		$\times$ & $\checkmark$ & $\checkmark\big / \times$ & $\checkmark$ & \citearticle{article_eprint_url_code} \\
		$\times$ & $\checkmark$ & $\checkmark\big / \times$ & $\times$ & \citearticle{article_eprint_url} \\
		$\times$ & $\times$ & $\checkmark\big / \times$ & $\checkmark$ & \citearticle{article_url_code} \\
		$\times$ & $\times$ & $\checkmark\big / \times$ & $\times$ & \citearticle{article_url} \\
	\end{tabular}
	
	Note that in particular citations via a URL alone are not recommended. If you want to cite a website or code repository, please use the respective reference classes \texttt{website} or \texttt{repository} (see below).
	
	\bibliographystylearticle{quantum}
	\bibliographyarticle{quantum_bst_demo}
	
	\pagebreak
	\section{Reference class \texttt{repository}}
	For the custom \texttt{repository} reference class, the \texttt{author} field is used if given but is not required (in contrast to the \texttt{article} class).
	If the repository address is given via \texttt{code} (strongly recommended), a properly formatted repository name is printed and links to the given address, including potentially version-, branch- or even commit-specific links.
	If no \texttt{code} entry is given, \texttt{url} is used as address instead, without any formatting of the printed text; Either \texttt{code} or \texttt{url} have to be provided.
	A title is not considered even if given.
	TODO: Consider a year in any way?
	
	\begin{tabular}{ccc}
		\texttt{code}& \texttt{url} & result \\
		$\checkmark$ & $\checkmark\big / \times$ &\citerepo{repo_code_url} \\
		$\times$ & $\checkmark$ &\citerepo{repo_url} \\
		$\times$ & $\times$ & invalid \\
	\end{tabular}
	
	Note that if you want both a \texttt{url} and a \texttt{code} link to be displayed, you can use the \texttt{website} reference class presented below for that.
	
	\bibliographystylerepo{quantum}
	\bibliographyrepo{quantum_bst_demo}
	
	\section{Reference class \texttt{website}}
	For the new custom reference class \texttt{website}, we require a \texttt{title} and a \texttt{url} which are both printed always.
	\texttt{author} is optional and printed if given, the same holds for \texttt{code}, which is formatted as repository link like for \texttt{repository}. If you want to provide \texttt{code} but not \texttt{url}, the reference class \texttt{repository} (see above) is made for you.
	
	\begin{tabular}{ccc}
		\texttt{author} & \texttt{code} & result \\
		$\checkmark$ & $\checkmark$ &\citeweb{web_author_code} \\
		$\times$ & $\checkmark$ &\citeweb{web_code} \\
		$\checkmark$ & $\times$ &\citeweb{web_author} \\
		$\times$ & $\times$ &\citeweb{web} \\
	\end{tabular}
	
	Note that if you want both a \texttt{url} and a \texttt{code} link to be displayed, you can use the \texttt{website} reference class presented below for that.
	
	\bibliographystyleweb{quantum}
	\bibliographyweb{quantum_bst_demo}
	
	\section{Tests}
	Directly from the arxiv~\citetests{hubregtsen2021training}, arxiv via Zotero~\citetests{hubregtsen2021training_2}, some more testcases~\citetests{holevo2012quantum,Holevo_2012,akers2020simple,katariya2021geometric,katariya2021geometric_2}
	
	\bibliographystyletests{quantum}
	\bibliographytests{quantum_bst_demo}
	
\end{document}
