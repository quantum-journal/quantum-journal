\documentclass[12pt]{quantumarticle}
\usepackage{amssymb,amsmath}

\usepackage[
  bookmarks=true,
  colorlinks,
  linkcolor=blue,
  urlcolor=blue,
  citecolor=blue,
  plainpages=false,
  pdfpagelabels,
  final,
  breaklinks=true
]{hyperref}



\begin{document}



\title{A sample \textit{Quantum} journal article}
\author{M. Muster}
\date{\today}
\maketitle
\abstract{
This is a sample article using the \texttt{quantumarticle} class, created for the journal \href{http://quantum-journal.org/}{\textit{Quantum}}. Authors are encouraged to use this document class for typesetting the final version. The class is available under the \href{https://latex-project.org/lppl/}{Latex Project Public License} and can be used widely.
}

\section{The \texit{Quantum} journal}
We propose the launch of an arXiv overlay journal for quant-ph. \textit{Quantum} will be a free and open access publishing option, providing high visibility for quality research on quantum science and related fields. It is an effort by researchers and for researchers to make science more open and publishing more transparent and efficient.

\section{Mission}
\textit{Quantum} is a free and open access peer-reviewed journal that provides high visibility for quality research on quantum science and related fields. It is an effort by researchers and for researchers to make science more open and publishing more transparent and efficient.

It addresses the growing dissatisfaction in the community with traditional, profit driven and impact factor focused models of scientific publishing, and their disproportionate effect on academics’ careers and the recent call for immediate open access publishing by the European Council.

\textit{Quantum} will provide a rigorous curating and peer-reviewing service for pre-prints in the arXiv section quant-ph. As an arXiv overlay journal, \textit{Quantum} will not directly host or print papers. Instead, a specific version of the paper on the arXiv will be officially accepted and listed on the website of \textit{Quantum}. Just like in conventional journals, all works receive a DOI upon acceptance and are then included in the standard citation counting databases.

\textit{Quantum} will publish high quality research, both theoretical and experimental. The main editorial criteria are correctness, relevance, and clarity. By focusing on the essential, \textit{Quantum} dramatically speeds up the publication process, reducing costs and hassle for authors, referees, and editors. For exceptional publications, \textit{Quantum} will feature editorial highlights and publicity through traditional and social media.

\textit{Quantum} is part of an increasing number of community-driven arXiv overlay journals, with examples in the fields of discrete analysis, computer science, mathematical physics, and astrophysics.


\section{Editorial criteria and peer review}
\paragraph{What:} Both original research and review articles, theoretical and experimental. Works should make a significant conceptual or technical contribution, but are not required to aim at a wide interdisciplinary audience.

\paragraph{Acceptance criteria:} Technical correctness, sound motivation of the work, significance, and clarity of presentation.  \textit{Quantum} encourages submissions that provide an honest assessment of their scope and limitations.

\paragraph{Review process:} Submissions are peer reviewed by at least two referees. Editorial rejection should only happen on scientific grounds. Incremental refereeing is discouraged: referees will be asked to either accept, reject, or accept under clearly spelled out conditions in the first round. \textit{Quantum} may offer optional dual consent open review in the future.

\section{Acceptance and publication}
\paragraph{Upon acceptance:} Authors upload an approved final version of the manuscript to the arXiv, which becomes the published version of the paper. They may opt in or out of a voluntary publication fee.

\paragraph{Template:} Authors are encouraged to use the \textit{Quantum} document class for typesetting the final version, ensuring a consistent look while providing maximal compatibility with existing LaTeX document classes. This is however not mandatory.

\paragraph{DOI:} The accepted version is given a DOI and is announced in \textit{Quantum}. Publications in \textit{Quantum} can be cited by volume and article number, and will be listed in the webofscience, so that citations count for citation metrics.

\paragraph{Publicity:} If provided by the authors, a non-technical abstract is also released by \textit{Quantum} and publicized in social media. For exceptional publications, the editor may ask a referee or external expert to write a short viewpoint about the paper.





\section{References}
Michael Downes \emph{Short Math Guide for \LaTeX}, AMS, 2002\\[0.2in]
George Gratzer, \emph{First Steps in \LaTeX}, Springer-Verlag, New York, 1999\\[0.2in]


\end{document}