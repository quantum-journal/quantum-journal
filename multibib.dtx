% \CheckSum{538}
%
% \iffalse
%% Package `multibib' to use with LaTeX2e.
%%
%% Copyright (C) 2008 by Thorsten Hansen
%%
%
% This work may be distributed and/or modified under the conditions of
% the LaTeX Project Public License, either version 1.3 of this license
% or any later version. The latest version of this license is in
% http://www.latex-project.org/lppl.txt and version 1.3 or later is
% part of all distributions of LaTeX version 2003/12/01 or
% later.
%
% This work has the LPPL maintenance status "maintained". The Current
% Maintainer of this work is Thorsten Hansen.
%
% This is a contributed file to the LaTeX2e system.
%
%
%<*dtx>
          \ProvidesFile{multibib.dtx}
%</dtx>
%<package>\NeedsTeXFormat{LaTeX2e}
%<package>\ProvidesPackage{multibib}
%<driver>\ProvidesFile{multibib.drv}
% \fi
%         \ProvidesFile{multibib.dtx}
          [2008/12/10 v1.4 Multiple bibliographies for one document.]
%
% \iffalse
%<*driver>
\documentclass{ltxdoc}
\usepackage{ifthen}
%
% Sort `General� before macros starting with `@'.
%
\makeatletter
\def\changes@#1#2#3{%
\protected@edef\@tempa{\noexpand\glossary{#1\levelchar
                                \ifx\saved@macroname\@empty
                                  \space
                                  \actualchar
                                  \generalname
                                \else
                                  \space                   %  <-- \space added
                                  \expandafter\@gobble
                                  \saved@macroname
                                  \actualchar
                                  \string\verb\quotechar*%
                                  \verbatimchar\saved@macroname
                                  \verbatimchar
                                \fi
                                :\levelchar #3}}%
 \@tempa\endgroup\@esphack}
\makeatother
%
%
%                    
\CodelineIndex
\RecordChanges
%\OnlyDescription
\begin{document}
\DocInput{multibib.dtx}
\end{document}
%</driver>
% \fi
%
% %%%%%%%%%%%%%%%%%%%%%%%%%%%%%%%%%%%%%%%%%%%%%%%%%%%%%%%%%%%%%%%%%%%%%%%%%%%%%
%
% \newcommand{\class}[1]{\textsf{#1}}
% \newcommand{\option}[1]{\textsf{#1}}
% \newcommand{\komascript}{KOMA-\textsc{Script}}
% \newcommand{\package}[1]{\textsf{#1}}
% \newcommand{\macrott}[1]{\texttt{\symbol{'134}#1}}
% \renewcommand*\descriptionlabel[1]{\hspace\labelsep\option{#1}}
%
% ^^A
% ^^A A rule to separate figures from the text.
% ^^A (Modified code from bibtopic.dtx).
% ^^A
% \newcommand\toprule[1][]{^^A
%   \ifthenelse{\equal{#1}{}}{^^A
%     \noindent{\makebox[\textwidth][t]{\hrulefill}}^^A
%     }{%
%     \noindent{\makebox[\textwidth][t]{\hrulefill^^A
%               \raisebox{-.5ex}{ \small\ttfamily#1 }\hrulefill}}^^A
%     }%
%   \par\noindent}
%
% \newcommand\bottomrule{\noindent\rule[\baselineskip]{\textwidth}{.6pt}}
%
%
% %%%%%%%%%%%%%%%%%%%%%%%%%%%%%%%%%%%%%%%%%%%%%%%%%%%%%%%%%%%%%%%%%%%%%%%%%%%%%
%
% \hyphenation{bib-lio-gra-phy bib-lio-gra-phies}
%
% \GetFileInfo{multibib.dtx}
% \title{The \textsf{multibib} Package}
% \author{Thorsten Hansen\\
%         \normalsize Thorsten.Hansen@psychol.uni-giessen.de}
% \date{\fileversion\quad \filedate}
% \maketitle
%
% %%%%%%%%%%%%%%%%%%%%%%%%%%%%%%%%%%%%%%%%%%%%%%%%%%%%%%%%%%%%%%%%%%%%%%%%%%%%%
%
% \GlossaryPrologue{\section*{{Change History}}%
%                   \markboth{{Change History}}{{Change History}}%
%                   Version numbers suffixed with characters have not
%                   been released.} 
% \def\levelchar{-}
%
% \changes{v1.0}{1999/07/27}{Initial version}
% \changes{v1.1}{1999/01/14}{Documentation.}
% \changes{v1.1a}{2001/10/06}{Equivalents of macros \macrott{citep} and 
%                             \macrott{citet} added.}
% \changes{v1.1b}{2001/10/10}{Evoke \package{multibib}'s \macrott{@citex} 
%                             variant not for standard \macrott{cite} 
%                             commands.}
% \changes{v1.1c}{2001/10/16}{Equivalents of macros \macrott{citealp} and
%                             \macrott{citealt} added.}
% \changes{v1.1d}{2001/11/04}{Spurious letters occurring for bibstyle
%                             \textsf{alpha} removed.}
% \changes{v1.1e}{2001/11/04}{Additional definition of \macrott{refname} 
%                             in \macrott{newcites} loop.}
% \changes{v1.2}{2003/03/24}{Documentation.}
% \changes{v1.2a}{2003/07/15}{General compatibility mechanism for packages
%                 which redefine \macrott{@citex}, motivated
%                 by compatibility to \package{cite} v4.0.}
% \changes{v1.2b}{2003/09/22}{General compatibility mechanism for packages 
%                 which define their own cite commands, motivated by 
%                 compatibility to \package{jurabib}.}
% \changes{v1.2c}{2004/01/23}{Code streamlined.}
% \changes{v1.3}{2004/01/24}{Documentation.}
% \changes{v1.3a}{2004/02/09}{Continuous numerical labels with natbib.}
% \changes{v1.3b}{2004/02/10}{\macrott{isnumber} changed to accept
%                 arguments with \macrott{fi}, e.g.,
%                 {\macrott{iffalse}{bjko}\macrott{fi}}.}  
% \changes{v1.3c}{2004/02/10}{Allow \macrott{bibliographystyle<s>} to
%                 be used in the preamble.} 
% \changes{v1.3d}{2005/10/25}{New option \option{labeled}.}
% \changes{v1.4}{2008/12/10}{Documentation.}
%                          
%
%
%
% \begin{abstract}
%   The \package{multibib} package allows to create references to
%   multiple bibliographies within one document. It thus provides a
%   complementary functionality to packages like \package{bibunits} or
%   \package{chapterbib}, which allow to create one bibliography for
%   multiple, but different parts of the document. The package
%   introduces the generic macro |\newcites| which generates
%   user-defined citation and bibliography commands for each new
%   bibliography in the document. The new commands can be used
%   intuitively, since both syntax and semantics are similar to their
%   standard \LaTeX\ equivalents. Each bibliography can have its
%   individual style and \BibTeX\ data file(s). Citations to each
%   bibliography are collected in a new auxiliary file to be processed
%   by \BibTeX. The \package{multibib} package is compatible with
%   \package{cite}, \package{inlinebib}, \package{jurabib} (to be
%   loaded before \package{multibib}), \package{natbib}, 
%   \package{suthesis} and \komascript\ classes. 
% \end{abstract}
%
%
%
%
% \section{Usage Notes}
%
% Suppose, you have to separate your citations into two bibliographies,
% one for primary literature and one for secondary literature. The
% \package{multibib} package defines the |\newcite| command which
% introduces a new family of cite and bibliography commands. 
% \begin{verbatim}
%   \newcites{sec}{Secondary Literature}
% \end{verbatim}
% In this case the generic macro defines four new commands,
% suffixed by the first argument.    
% \begin{verbatim}
%   \citesec  
%   \nocitesec
%   \bibliographystylesec  
%   \bibliographysec
% \end{verbatim}
% The second argument of |\newcites| provides the title of the corresponding
% bibliography. All these new commands accept the same parameters as
% their standard \LaTeX\ equivalents and behave analogous to.---In the
% example above, the bibliography for primary literature can be
% generated by \LaTeX's standard commands. When the document is
% processed, an auxiliary file |sec.aux| is generated, which needs to
% be compiled through \BibTeX.
%
% The tiny \verb|\newcites| command is not limited to one bibliography.
% In fact, you can generate as much bibliographies as you like (only
% limited by the maximum number of \TeX's output files, usually~16).
% In the following example, separate bibliographies are created for
% own articles, submitted articles, internal reports, and other work.
% \begin{verbatim}
%   \newcites{own,submitted,internal}%
%     {Own Work,%
%      Submitted Work,%
%      {Technical Reports, Master Theses and Ph.D. Theses}}
% \end{verbatim}
% Note that a title containing commas needs to be grouped. Again, the
% bibliography for other work is generated by \LaTeX's standard
% commands.
%
% \DescribeMacro{\newcites}
% After this introductory examples, let's look at the precise
% definition of |\newcites|. The command
% |\newcites|\marg{suffix\_list}\marg{heading\_list} defines a new
% family of commands for each suffix~\meta{s} in \meta{suffix\_list},
% namely |\cite|\meta{s}, |\nocite|\meta{s},
% |\bibliographystyle|\meta{s}, and |\bibliography|\meta{s} . The
% \meta{suffix\_list} is a comma separated list of letters, i.e, the
% same characters which are allowed as part of a \TeX\ command. The
% \meta{heading\_list} is a comma separated list of words, i.e, the
% same sort of input (including commands) that can be used as argument
% of, e.g., |\section|. The commands |\cite|\meta{s} and
% |\nocite|\meta{s} generate citations that appear in the
% corresponding bibliography which is generated by
% |\bibliography|\meta{s}. The title of this bibliography is defined
% by the entry \meta{h} in \meta{heading\_list} at the same position
% as \meta{s} in \meta{suffix\_list}. The individual style of the
% bibliography can be defined by |\bibliographystyle|\meta{s}.
% 
% \DescribeMacro{\setbiblabelwidth}
% When using numerical reference schemes like, e.g., \textsf{plain},
% the labels in the bibliography have sometimes the wrong width. In
% this case you can set the desired label width using
% |\setbiblabelwidth|\marg{number}, in the same fashion as you would
% specify the label width with |\thebibliography|\marg{number}. 
%
% It is also possible to let \LaTeX\ do the job by using
% \package{multibib}'s variant \textsf{mbplain} instead of
% \textsf{plain} as bibliography style. The modification does not
% affect the formatting style but only internal values, such that
% \LaTeX\ can determine the right width. More precisely,
% \textsf{mbplain} sets the longest label to the number of entries in
% the bibliography, while \textsf{plain} takes the smallest number
% which has the same width as the number of entries. For example, if
% your bibliography has 29~entries, \textsf{plain} sets the label to
% 10 while \textsf{mbplain} sets the label to~29. This is done by
% modifying the function \texttt{longest.label.pass}.
% \begin{verbatim}
% FUNCTION {longest.label.pass}
% { number.label int.to.str$ 'label :=
%   number.label #1 + 'number.label :=
%   label width$ longest.label.width >
%   label width$ longest.label.width =   %% added in mbplain
%   or                                   %% added in mbplain
%     { label 'longest.label :=
%       label width$ 'longest.label.width :=
%     }
%     'skip$
%   if$
% }
% \end{verbatim}
% If you use other numerical styles than \textsf{plain}, you can
% customize your favorite style accordingly.
%
%
%
% \subsection{Package Options}
% \begin{description}
% \item[labeled] Add the internal suffixes for each
%      bibliography as prefixes to the labels to the bibliography. For
%      example, if you define  
% \begin{verbatim}\newcites{A,B,C}%
%   {Own \LaTeX\ Work,%
%   Submitted,%
%   {Habilitations, Dissertations and Diploma Thesis}}
% \end{verbatim}
%      and specify a numerical bibstyle
%      \verb+\bibliographystyleA{plain}+, the labels in the
%      bibliography and the cited references will be [A1], [A2], [A3],
%      \dots; analogously, \verb+\bibliographystyleB{plain}+ results
%      in labels and cited references [B1], [B2], [B3],\dots; and
%      \verb+\bibliographystyleC{plain}+ results in labels and cited
%      references [C1], [C2], [C3], \dots . This option is useful
%      together with option \option{resetlabels} and numerical
%      reference schemes.
%
% \item[resetlabels] When using numerical reference schemes,
%      start each bibliography with~`[1]'. Default is continuous numbering, 
%      such that labels are created unambiguously.
% \end{description}
% 
%
%
% \subsection{\textnormal{\BibTeX} Processing}
%
% For each bibliography a corresponding auxiliary file \meta{s}|.aux|
% is generated, that needs to be compiled through \BibTeX. This can be
% done by the following bash-script.
% \begin{verbatim}
%    #!/bin/bash
%    for file in *.aux ; do
%      bibtex `basename $file .aux`
%    done
% \end{verbatim}
% 
%
%
% \subsection{Limitations}
%
% \begin{itemize}
% \item When numerical reference schemes are used and references to the
%       same item appear in different bibliographies, the labels
%       cannot be resolved properly. Rather, the replacement text
%       (i.e, the number) is used which is defined in the auxiliary
%       file read last.
% \item When using author-year reference schemes, entries with the
%       same author and year combinations which appear in different
%       bibliographies get the same label.
%       In this case you have to manually add some hints to which
%       bibliography the entry refers, e.g.,\\ 
%       |\cite[in primary literature]{Foo:1999a}| vs.\ \\  
%       |\citesec[in secondary literature]{Foo:1999b}|.\\
%        You can also edit the |.bbl| files created by \BibTeX\ and
%       add `a', `b',\dots\ to the labels, but your changes will
%       be overwritten by subsequent \BibTeX\ runs. The
%       differentiating letters `a', `b',\dots\ cannot be added
%       automatically by \BibTeX\ since the entries appear in
%       different auxiliary files.
% \item With \package{natbib}, you cannot use numerical and
%       author-year reference schemes together. Typographically, this
%       is regarded bad style anyway.
% \item For \package{jurabib} and \package{multibib}, the order of
%       loading is important: \package{jurabib} must be loaded before 
%       \package{multibib}.
% \end{itemize}
%
%
%
%
% \section{Example}
% 
% Let's now consider a complete example with two bibliographies, one
% for \TeX\ and \LaTeX\ references and one for Postscript references.
% For demonstration purposes, different styles are used for the two
% references, namely \textsf{alpha} for \TeX\ and \LaTeX\ references and
% \textsf{plain} for Postscript references. Typographically, different
% bibliography styles are regarded as bad and therefore should be
% avoided in real documents. The reference data are supposed to be
% stored in a common \BibTeX\ data file |lit.bib|. Note that the
% heading of the Postscript references is redefined with
% |\renewcommand{\refname}|.

% \begin{figure}[h]
% \hspace*{-25mm}\begin{minipage}[t]{0.6\textwidth}
% \toprule
% {\let\small\footnotesize
% \begin{verbatim}
% \documentclass{article}
% \usepackage{multibib}
% \newcites{ltex}{\TeX\ and \LaTeX\ References}
% 
% \begin{document}
% References to the \TeX book \citeltex{Knuth:1991} 
% and to Lamport's \LaTeX\ book, which appears 
% only in the references\nociteltex{Lamport:1994}. 
% Finally a cite to a Postscript tutorial
% \cite{Adobe:1985}.
% 
% \bibliographystyleltex{alpha}
% \bibliographyltex{lit}
% 
% \renewcommand{\refname}{Postscript References}
% \bibliographystyle{plain}
% \bibliography{lit}
% 
% \end{document}
% \end{verbatim}}
% \bottomrule
% \end{minipage}
% \hspace{5mm}
% \begin{minipage}[t]{0.65\textwidth}
% \toprule
% \leftmargini1mm
% \begin{quote}
%   \footnotesize
%   References to the \TeX book \cite{Knuth:1991} 
%   and to Lamport's \LaTeX\ book, which appears 
%   only in the references\nocite{Lamport:1994}. 
%   Finally a cite to a Postscript tutorial
%   \cite{Adobe:1985}.
%   {\let\section\subsubsection
%   \renewcommand{\refname}{\TeX\ and \LaTeX\ References}
%   \begin{thebibliography}{Lam94}
%   \bibitem[Knu91]{Knuth:1991}
%   Donald~E. Knuth.
%   \newblock {\em The {\TeX} book}.
%   \newblock Addison-Wesley, Reading, Massachusetts, 1991.
%   \bibitem[Lam94]{Lamport:1994}
%   Leslie Lamport.
%   \newblock {\em {\LaTeX:} A Document Preparation System}.
%   \newblock Addison-Wesley, Reading, Massachusetts, 2 edition, 1994.
%   \end{thebibliography}}
%   {\let\section\subsubsection
%   \renewcommand{\refname}{Postscript References}
%   \begin{thebibliography}{1}
%   \bibitem{Adobe:1985}
%   {Adobe {S}ystem {I}ncorporated}.
%   \newblock {\em Postscript Language Tutorial and Cookbook}.
%   \newblock Addison-Wesley, Reading, Massachusetts, 1985.
%   \end{thebibliography}}
% \end{quote}
% \vspace{3.25mm}
% \bottomrule
% \end{minipage}
% \caption{Example input and output.}
% \end{figure}
% 
% To process your document, three runs of \LaTeX\ and two runs of \BibTeX\
% are required.
%
% \vbox{%
% \begin{verbatim}
%   latex mydoc
%   bibtex mydoc
%   bibtex ltex
%   latex mydoc
%   latex mydoc
% \end{verbatim}}
%
%
%
%
% \section{Notes for Class and Package Writers}
% 
% Packages such as \package{cite} or \package{natbib} which modify the
% internal macro |\@citex| called from |\cite| by, e.g., changing
% the number of arguments, and/or define new variants of |\cite|, such as
% \package{natbib}'s |\citep|, deserve a special handling to ensure
% compatibility to \package{multibib}. In the first case, the problem
% arise because \package{multibib}'s redefinition of |\@citex| to write to
% a new auxiliary file |\@newciteauxhandle| instead of writing to
% |\@auxout| is overwritten. In the second case, the problem arise
% because \package{multibib} is unaware of additional cite commands and
% thus cannot define the new variants. The problems can
% be solved by using the compatibility mechanisms
% supported by \package{multibib} as described in the following.
% 
%
%
% \subsection{Packages which modify \macrott{@citex}}
% 
% Packages which modify |\@citex| can be made compatible to
% \package{multibib} by making the following three changes to the code:
% \begin{enumerate}
% \item Add the following code:
%   \begin{verbatim}
%   \providecommand\@newciteauxhandle{\@auxout}
%   \def\@restore@auxhandle{\gdef\@newciteauxhandle{\@auxout}}
%   \AtBeginDocument{%
%     \@ifundefined{newcites}{\global\let\@restore@auxhandle\relax}{}}
%   \end{verbatim}
% The first line defines |\@newciteauxhandle| to the default value
% |\@auxout| if |\@newciteauxhandle| is not already defined. The
% second line defines macro |\@restoreauxhandle| which sets
% |\@newciteauxhandle| to its default value. The next two lines
% |\let| macro |\@restoreauxhandle| to |\relax| in case
% \package{multibib} is not loaded and |\newcites| is undefined.
% \item In the redefinitions of |\@citex| replace |\@auxout| by
%   |\@newciteauxhandle|.
% \item At the end of the each |\@citex| redefinitions add the macro
%   |\@restoreauxhandle|.
% \end{enumerate}
% At present, packages \package{cite} and \package{jurabib} use this
% compatibility mechanism.
% 
%
%
% \subsection{Packages which define new cite commands}
% 
% Packages which define new cite commands can add these commands using
% |\@mb@citenamelist|. The default definition, which
% already includes \package{natbib}'s cite variants, is given below.
% \begin{verbatim}
%   \def\@mb@citenamelist{cite,citep,citet,citealp,citealt}
% \end{verbatim}
% A package which defines, e.g, a new cite command |\footcite| informs
% \package{multibib} about this command by defining
% \begin{verbatim}
%   \def\@mb@citenamelist{cite,citep,citet,citealp,citealt,footcite}
% \end{verbatim}
% At present, package \package{jurabib} use this compatibility mechanism.
%
%
%
% \subsection{Advantages of the compatibility mechanism}
%
% Compared to a solution where all compatibility patching is done by
% \package{multibib}, such as is presently the case for
% \package{natbib}, the above solution has several advantages: 
% \begin{enumerate}
% \item The mechanisms provides a general solution for several
%   packages. Any new package can be made compatible with \package{multibib}
%   without the need for a new release of \package{multibib}.
% \item Existing packages can change the number of arguments to
%   |\@citex| or introduce new cite macros without the need to change
%   \package{multibib}.
% \item If |\@citex| needs to read the next input character, this
%   character is available and not masked by \package{multibib}'s |let|s
%   which follow the |\@std@citex|. This is important for the
%   \package{cite} package.
% \item More efficient code. A single |\let| in |\@restore@auxhandle|
%   instead of four.
% \end{enumerate}
%
%
%
%
% \section{Brief Discussion of Related Packages}
% 
% \paragraph{Similar packages.} The \package{bibtopic} package addresses the
% same issue as the present package, using a completely different
% approach. The user has to provide a separate \BibTeX\ data file
% |liti.bib| for each bibliography and can generate the corresponding
% bibliography within a |btSect| environment.
%
% \begin{verbatim}
% \begin{btSect}{liti}
%   \btPrintCited
% \end{btSect}
% \end{verbatim}
%
% Using |\btPrintNotCited|, a complementary bibliography showing all
% entries not cited in the document can also be generated.
% Additionally, an environment |btunit| is defined to generate
% bibliographies for different units of the text, thus partly
% offering the functionality of packages like \package{bibunits} or
% \package{chapterbib}. Besides its powerful functionality, the
% \package{bibtopic} package has some drawbacks and restrictions:
% \begin{itemize}
% \item less intuitive user interface
% \item the user has to split his \BibTeX\ data files
% \item bibliography styles that put some material between the begin
%       of\\ |thebibliography| and the first |\bibitem| deserve a
%       special handling
% \item does not work with bibliography style |unsorted| 
% \item and more, see Sec.~`Restrictions' of the \package{bibtopic}
%       documentation
% \end{itemize}
% 
% Finally, due to its large size (1059 lines) and implementation
% approach, processing is probably slower. In a nutshell,
% \package{bibtopic} does the following: for each bibliography, a
% |\citation{*}| command is written to a new, corresponding auxiliary
% file, such that \emph{all} entries of the \BibTeX\ data base are
% extracted. Then, by reimplementing the |\bibitem| command, all items
% which do not occur in a special list of labels, produced by the cite
% commands, are deleted. Obviously, doing part of \BibTeX's work
% requires some amount of memory and time (and programming skills,
% noteworthy |;-)|).
%
% 
% \paragraph{Complementary packages.} The packages \package{bibunits}
% and \package{chapterbib} provide commands to generate multiple
% bibliographies, too. The main difference is, that with these
% packages a single bibliography is generated for multiple, but
% different parts of the text. Therefore, within a certain unit, i.e.,
% part of the text, citations can only be made to a single,
% corresponding bibliography. On the other hand, with
% \package{multibib}, within the same (but single) part, citations can
% be generated for multiple bibliographies. 
%
% \paragraph{Analogous packages.} The \package{multind} package allows
% multiple indexes.
%    
%
%
%
% \section{Acknowledgments} 
%
% The author thanks Donald Arseneau for suggesting the compatibility mechanism
% for packages which modify |\@citex|. Further, the author
% acknowledges the contributions of numerous people, in particular
% Jens Berger and Frank Mittelbach, whose suggestions and bug reports have 
% helped to improve \package{multibib}.
%
%
% 
%
% \StopEventually{\PrintIndex\PrintChanges}
%
%
%
%
% \section{The Macros}
%
% One paradigm which guided the development of the code was to use
% standard \LaTeX\ commands as much as possible, and to customize their
% behavior using |\let|.
%
% 
%    \begin{macrocode}
%<*package>
%    \end{macrocode}
%
%
% \subsection{Option Handling}
%    \changes{v1.0i}{2000/01/12}{New option \option{resetlabels}.}
%
% \begin{macro}{\ifcontinuouslabels}
%    Define a new if to switch between continuous labeling of
%    bibliographies (default) and start of labels with~`[1]'. The latter 
%    can be activated with option \option{resetlabels}.
%    \begin{macrocode}
\newif\ifcontinuouslabels 
\continuouslabelstrue
\DeclareOption{resetlabels}{\continuouslabelsfalse} 
%    \end{macrocode}
% \end{macro}
%
%
% \begin{macro}{labeled}
%    \changes{v1.3d}{2005/10/25}{Define option \option{labeled}.}
%
%    Define a new option to add the internal suffixes for each
%    bibliography as labels to the bibitems.
%    \begin{macrocode}
\newif\iflabeled
\labeledfalse
\DeclareOption{labeled}{\labeledtrue}
%    \end{macrocode}
% \end{macro}
%
%
%
% \noindent Finally, process all package options.
%
%    \begin{macrocode}
\ProcessOptions
%    \end{macrocode}
%   
%
% \subsection{Preliminaries}
%
% \begin{macro}{\mylop}
% \begin{macro}{\mylopoff}
%    Variation of |\lop|, using `|,|' as separator instead of `|\|'
%    to extract the elements in the first argument of |\newcites|
%    (cf.~The \TeX book, p. 378).  
%    \begin{macrocode}
\def\mylop#1\to#2{\expandafter\mylopoff#1\mylopoff#1#2}
\long\def\mylopoff#1,#2\mylopoff#3#4{\def#4{#1}\def#3{#2}}
%    \end{macrocode}
% \end{macro}
% \end{macro}
%
%
% \begin{macro}{\@newciteauxhandle}
%    \changes{v1.1b}{2001/10/10}{Handle removed.}
%
%    Define a new handle of the auxiliary file for all |\cite| and
%    |\nocite| commands, standard as well as newly defined. For the
%    standard commands, this handle is let to |\auxout|. Since version
%    1.1b the definition |\let\@newciteauxhandle\@auxout| is no longer
%    needed since  below we reset |\@citex| to its standard value
%    after each call of |\mb@@citex|. 
% \end{macro}
%
%
% \begin{macro}{\std@@citex}
% \begin{macro}{\mb@@citex}
%    \changes{v1.0f}{2000/01/10}{Update to \package{natbib} v7.0:
%                                save \macrott{@auxout} before
%                                redefinition instead of encapsulation
%                                in braces.}    
%    \changes{v1.1b}{2001/10/10}{\macrott{global} setting of 
%             \macrott{@save@auxout}; setting of \macrott{newciteauxhandle}
%             removed; instead, reset \macrott{citex} to its standard 
%             version.}
%    \changes{v1.2a}{2003/07/15}{New definition of \macrott{mb@@citex}
%             only if \macrott{@newciteauxhandle} is not already
%             defined by compatible packages.}
%                    
%
%    Definitions of |\mb@@citex| to replace |\@citex| in the definition of     
%    the various |cite| commands. The idea is to write to a specific file
%    referred to |\@newciteauxhandle| instead of |\@auxout|.  Below in the
%    definition of the specific |\cite<s>| and |\nocite<s>| commands,
%    |\@newciteauxhandle| is set to the new auxiliary files |<s>.aux|. If
%    |\@newciteauxhandle| is already defined by compatible packages, simply
%    |\let| macro |\mb@@citex| to |\@citex|
%  
%    If \package{natbib} is loaded, a variant of |\@citex| with an 
%    additional optional argument is needed. Macro |\NAT@set@cites|
%    invokes \package{natbib}'s |\cite| and |\@citex| definitions.
%    \begin{macrocode}
\AtBeginDocument{%
  \@ifpackageloaded{natbib}{%
    \NAT@set@cites
    \let\std@@citex\@citex
    \def\mb@@citex[#1][#2]#3{%
      \global\let\@save@auxout\@auxout
      \let\@auxout\@newciteauxhandle 
      \std@@citex[#1][#2]{#3}%
      \let\@auxout\@save@auxout
      \let\@citex\std@@citex}%
  }{% else
    \@ifundefined{@newciteauxhandle}{% not defined by compatible packages
      \let\std@@citex\@citex
      \def\mb@@citex[#1]#2{{%
        \let\@auxout\@newciteauxhandle
        \std@@citex[#1]{#2}}%
        \let\@citex\std@@citex}%
      }{\let\mb@@citex\@citex}%
  }%
}%
%    \end{macrocode}
% \end{macro}
% \end{macro}
%
%
% \begin{macro}{\@newusecounter}
% \begin{macro}{\newusecounter}
%    Do not reset counter to zero at begin of bibliography. This
%    generates continuous numbering of references through all new
%    bibliographies to ensure that numerical references are created
%    unambiguously. To reset the counter only for the \emph{first}
%    bibliography, |\newusecounter| is initially let to |\usecounter|.
%    Below, in the definition of the new bibliographies,
%    |\newusecounter| is let to |\@newusecounter|.
%    \begin{macrocode}
\def\@newusecounter#1{\@nmbrlisttrue\def\@listctr{#1}}
\let\newusecounter\usecounter
%    \end{macrocode}
% \end{macro}
% \end{macro}
%
%
% \begin{macro}{\std@bibliography}
% \begin{macro}{\bibliography}
%    \changes{v1.0a}{1999/07/27}{Continuous numbering also for 
%                                \macrott{bibliography}.}
%    \changes{v1.0i}{2000/01/12}{Change to respond to option
%                                \option{resetlabels}.} 
%
%    To ensure continuous numbering, the |\newusecounter| is
%    redefined also for the standard |\bibliography|, same as for the
%    new |\bibliography|\meta{s} macros.
%    \begin{macrocode}
\let\std@bibliography\bibliography
\def\bibliography#1{%
  \ifcontinuouslabels
    \let\usecounter\newusecounter
  \fi
  \std@bibliography{#1}%
  \ifcontinuouslabels
    \global\let\newusecounter\@newusecounter
  \fi}
%    \end{macrocode}
% \end{macro}
% \end{macro}
%
%
% \begin{macro}{\mb@biblabelwidth}
% \begin{macro}{\setbiblabelwidth}
%    \changes{v1.0e}{1999/12/12}{New macro.}
%
%    The macro |\setbiblabelwidth| explicitly sets the width of the
%    labels in the bibliography. An internal counter
%    |\mb@biblabelwidth| is introduced and a macro |\setbiblabelwidth|
%    which simply assigns its argument to the new counter. The value
%    of the counter is then used in \package{multibib}'s modification
%    of |\thebibliography|.
%    \begin{macrocode}
\newcount\mb@biblabelwidth
\newcommand\setbiblabelwidth[1]{\mb@biblabelwidth #1}
%    \end{macrocode}
% \end{macro}
% \end{macro}
%
%
% \begin{macro}{\std@thebibliography}
% \begin{macro}{\thebibliography}
%    \changes{v1.0b}{1999/07/30}{Correct label width for numerical
%                                reference schemes.}
%    \changes{v1.0e}{1999/12/12}{Works with \macrott{setbiblabelwidth}.}
%    \changes{v1.0i}{2000/01/14}{No change of label width for option
%                                \option{resetlabels}.} 
%    \changes{v1.0k}{1999/01/14}{\package{suthesis} compatibility.}
%    \changes{v1.3a}{2004/02/09}{Natbib specific definition of
%    thebibliography, modified to use \macrott{usecounter}.}
%
%    The argument of |\thebibliography| determines the width of the
%    labels in the bibliography. First it is checked if the parameter
%    is a number. If it is not a number, we do not have to care about
%    setting label widths and can simply call |\std@thebibliography|.
%    Otherwise we check if the label width is not set explicitly,
%    i.e., |\mb@biblabelwidth=0|. In this case the label width is determined
%    as the the last label in the previous bibliography |\c@enumiv|
%    plus the number of labels in the current bibliography |#1|
%    (provided the bibliography style computes the values right,
%    like \textsf{mbplain}). If the value is set explicitly
%    by~|\mb@biblabelwidth|, this value is used and then reset to zero
%    such that only the labels in the current bibliography are affected.
%
%    \begin{macrocode}
\AtBeginDocument{%
  \@ifpackageloaded{suthesis}%
    {\def\thebibliography#1{%
       \newpage
       \addcontentsline{toc}{chapter}{\bibname}%
       \@ldthebibliography{#1}}}%
    {}%
  \@ifpackageloaded{natbib}%
    {%


\renewenvironment{thebibliography}[1]{%
 \bibsection\parindent \z@\bibpreamble\bibfont\list
   {\@biblabel{\arabic{NAT@ctr}}}{\@bibsetup{#1}%
    \usecounter{NAT@ctr}}% %% only changed here to usecounter
    \ifNAT@openbib
      \renewcommand\newblock{\par}
    \else
      \renewcommand\newblock{\hskip .11em \@plus.33em \@minus.07em}%
    \fi
    \sloppy\clubpenalty4000\widowpenalty4000
    \sfcode`\.=1000\relax
    \let\citeN\cite \let\shortcite\cite
    \let\citeasnoun\cite
 }{\def\@noitemerr{%
  \PackageWarning{natbib}
     {Empty `thebibliography' environment}}%
  \endlist\vskip-\lastskip}

      \let\std@thebibliography\thebibliography
      \def\thebibliography#1{%
        \@isnumber{#1}%
        {\ifnum\mb@biblabelwidth=0
          \@tempcnta\c@NAT@ctr %% changed here to c@NAT@ctr
          \ifcontinuouslabels
          \advance\@tempcnta#1%
          \fi
          \std@thebibliography{\@arabic\@tempcnta}%
          \else
          \std@thebibliography{\@arabic\mb@biblabelwidth}%
        \global\mb@biblabelwidth 0
        \fi}%
        {\std@thebibliography{#1}}%
     }%
   }%
   {% else, natbib not loaded
     \let\std@thebibliography\thebibliography

     \def\thebibliography#1{%
        \@isnumber{#1}%
          {\ifnum\mb@biblabelwidth=0
            \@tempcnta\c@enumiv
            \ifcontinuouslabels
            \advance\@tempcnta#1%
            \fi
            \std@thebibliography{\@arabic\@tempcnta}%
            \else
            \std@thebibliography{\@arabic\mb@biblabelwidth}%
          \global\mb@biblabelwidth 0
          \fi}%
        {\std@thebibliography{#1}}%
    }%
   }%
}
%    \end{macrocode}
% \end{macro}
% \end{macro}
%
%
% \begin{macro}{\@isnumber}
%    \changes{v1.0g}{2000/01/10}{Empty string is not handled as number.}
%    \changes{v1.1d}{2001/01/04}{Execute \macrott{@scannumber} in
%                                \macrott{hbox}.}
%    \changes{v1.3b}{2004/02/10}{\macrott{isnumber} changed to accept
%                                arguments with \macrott{fi}, e.g.,
%                                {\macrott{iffalse}{bjko}\macrott{fi}}.}  
%
%    Macro |\@isnumber| behaves as follows: If |#1| is a number (which
%    may contain blanks at arbitrary positions), |#2| is executed,
%    else |#3|.
%
%    \begin{macrocode}
\newcommand{\@isnumber}[3]{%
  \def\argdef{#1}%
  \edef\argedef{#1}%
  \ifx\argedef\empty
    #3
  \else
    \ifx\argdef\argedef
      \global\@tempswafalse
      \setbox\@tempboxa=\hbox{\@scannumber#1\plugh}%
      \if@tempswa#2\else#3\fi
    \else
      #3
    \fi
  \fi}
%    \end{macrocode}
% \end{macro}
%
%
% \begin{macro}{\@scannumber}
%    \changes{v1.1d}{2001/01/04}{Initialize \macrott{plugh} with
%                                \macrott{relax}; simply \macrott{relax} 
%                                when the \macrott{testchar} is not a digit 
%                                instead of \macrott{@gobbletillplugh}.} 
%
%    Macro |\@scannumber| evaluates the single characters
%    of its argument. If all characters are digits or
%    blanks,|\@scannumber| is recursively evoked until it finally  
%    reads |\plugh| which marks the end of the argument (in
%    macro |\@isnumber|) and sets |\@tempswatrue| which is then used
%    in |\@isnumber|. If the tested character is not a digit, the
%    macro exits, leaving |\@tempswa| unchanged.
%
%    \begin{macrocode}
\let\plugh\relax
\newcommand{\@scannumber}[1]{%
  \let\testchar#1%
  \ifx \testchar 0\let\next\@scannumber
  \else\ifx \testchar 1\let\next\@scannumber
  \else\ifx \testchar 2\let\next\@scannumber
  \else\ifx \testchar 3\let\next\@scannumber
  \else\ifx \testchar 4\let\next\@scannumber
  \else\ifx \testchar 5\let\next\@scannumber
  \else\ifx \testchar 6\let\next\@scannumber
  \else\ifx \testchar 7\let\next\@scannumber
  \else\ifx \testchar 8\let\next\@scannumber
  \else\ifx \testchar 9\let\next\@scannumber
  \else\ifx \testchar \plugh \let\next\relax \global\@tempswatrue
  \else \let\next\relax
  \fi\fi\fi\fi\fi\fi\fi\fi\fi\fi\fi
  \next}
%    \end{macrocode}
% \end{macro}
%
%
% \begin{macro}{\@gobbletillplugh}
%    \changes{v1.1d}{2001/01/04}{Macro removed.}
%
%    Macro |\@gobbletillplug| gobbles all input up to (and including)
%    |\plugh| by defining |\def\@gobbletillplugh#1\plugh{}|. Since
%    version 1.1d, |\@scannumber| is evaluated in a |\hbox| and the
%    gobbling is no longer needed.
% \end{macro}
%
%
% \begin{macro}{\mb@addtocontents}
%    \changes{v1.0d}{1999/08/19}{Add bibliography name to table of contents.}
%
%    Some styles like \package{suthesis} or \komascript\ classes like
%    \class{scrartcl} with option \option{bibtotoc} redefine
%    |\bibliography| such that the heading of the bibliography appears
%    in the table of contents. Entries for the table of contents are
%    generated by writing the appropriate information to |\@auxout|.
%    Because |\@auxout| is locally redefined when the bibliography is
%    read, entries go to the wrong file. Therefore, a macro
%    |\mb@addtocontents| is introduced which writes to |\temp@auxout|.
%    Below, in the definition of the new |\bibliography|\meta{s},
%    |\mb@addtocontents| replaces the standard |\addtocontents|, and
%    |\temp@auxout| is set to the default |\@auxout|.
%    \begin{macrocode}
\long\def\mb@addtocontents#1#2{%
  \protected@write\temp@auxout
    {\let\label\@gobble \let\index\@gobble \let\glossary\@gobble}%
    {\string\@writefile{#1}{#2}}}
%    \end{macrocode}
% \end{macro}
%
%
% \begin{macro}{\bibname}
%    \changes{v1.0h}{2000/01/10}{Changed to evoke \macrott{refname} instead.} 
%    \changes{v1.0j}{2000/01/13}{New implementation: Check if
%                                \macrott{bibname} is undefined
%                                instead of checking if class
%                                \class{book} or \class{report} is loaded.} 
%    \changes{v1.1e}{2001/01/04}{Macro removed.}
%
%    Package~\package{multibib} redefines the headings of the new
%    bibliographies using |\refname|. Some classes like \class{book} and
%    \class{report} use |\bibname| instead of |\refname|. Instead of
%    changing the definition of |\bibname|, we simply define
%    |\refname| in the same was as |\bibname| in the |newcites| loop below.
% \end{macro}
%
%
% \begin{macro}{\@mb@citenamelist}
%    \changes{v1.2b}{2003/09/22}{Macro added.}
%   
%    Define list of cite commands to be processed within the
%    |\newcites| loop below. If already defined by other packages,
%    |\relax|.
%    \begin{macrocode}
\@ifundefined{@mb@citenamelist}%
  {\def\@mb@citenamelist{cite,citep,citet,citealp,citealt}}%
  {\relax}
%    \end{macrocode}
% \end{macro}
%
%
%
% \subsection{The \texttt{newcites} loop}
%
% \begin{macro}{\newcites}
%    Loop over all headings in \meta{heading\_list}. The current
%    heading is stored in |\@newrefname|, the corresponding suffix is
%    stored in |\@suffix|.
%    \begin{macrocode}
\def\newcites#1#2{%
  \def\@suffixlist{#1,}%
  \@for\@newrefname:=#2\do{%
    \mylop\@suffixlist\to\@suffix
%    \end{macrocode}
% \end{macro}
%
% \begin{macro}{\@refname<s>}
%    Define the reference title |\protected|, such that the title can
%    contain control sequences as, e.g., in the title `\LaTeX\
%    References'. 
%    \begin{macrocode}
    \expandafter\protected@edef\csname refname\@suffix\endcsname
      {\@newrefname}%
%    \end{macrocode}
% \end{macro}
%
%
% \begin{macro}{\@auxout<s>}
% \begin{macro}{\@auxout<s>name}
%    \changes{v1.0c}{1999/08/19}{\macrott{makeatletter} when input new 
%                                auxiliary files.}
%
%    Define new write. Input the auxiliary file if it exists before
%    opening it, to read the replacement text for the |\cite| commands
%    which is generated by |\bibcite|. Because some styles (like
%    \package{inlinebib.sty}) write command names containing an |@| to 
%    the auxiliary files, |\makeatletter| is locally set.
%    \begin{macrocode}
    \if@filesw
      \expandafter\newwrite\csname @auxout\@suffix\endcsname
      \expandafter\edef\csname @auxout\@suffix name\endcsname{\@suffix}%
      \begingroup
        \makeatletter
        \@input@{\csname @auxout\@suffix name\endcsname .aux}%
      \endgroup
      \immediate\openout\csname @auxout\@suffix\endcsname
                        \csname @auxout\@suffix name\endcsname .aux
    \fi
%    \end{macrocode}
% \end{macro}
% \end{macro}
%
%    Define new |\cite| and |\nocite|. For |\nocite|, we simple let
%    |\@auxout| to the new auxiliary file. For |\cite|, this
%    approach does not work because |\cite| can have an optional
%    argument, so we cannot enclose the redefinition of |\@auxout| in
%    braces. We slightly modify |\@citex| (see above) to use a special
%    handle, |\@newciteauxhandle|, which is let to the new auxiliary file.
%
% \begin{macro}{\cite<s>}
%    \changes{v1.2b}{2003/09/22}{Loop over \macrott{@mb@citenamelist} added.}
%
%    For each \meta{citename} in |\@mb@citenamelist|, define
%    \package{multibib}'s new \meta{citename}\meta{s} commands. For
%    example, if \meta{citename} := |cite| and if \meta{s} := |own|, a command
%    |\citeown| is defined as:
%    \begin{verbatim}
%      \def\citeown{%
%        \let\@citex\mb@@citex
%        \let\@newciteauxhandle\@auxoutown
%        \cite}
%    \end{verbatim}
%    \begin{macrocode}
    \@for\@citename:=\@mb@citenamelist\do{%
      \expandafter\edef\csname \@citename\@suffix\endcsname{%
        \let\noexpand\@citex\noexpand\mb@@citex
        \let\noexpand\@newciteauxhandle\csname @auxout\@suffix\endcsname
        \noexpand\csname\@citename\endcsname}%
    }%
%
%    \end{macrocode}
% \end{macro}
%
%
% \begin{macro}{\nocite<s>}
%    \changes{v1.2c}{2004/01/23}{\macrott{let} \macrott{@citex}
%             \macrott{mb@@citex} removed.}
%
%    Define new |\nocite|\meta{s}, e.g., if \meta{s} := |own|, a
%    command |\nociteown| is defined as:
%    \begin{verbatim}
%      \def\nociteown##1{{%
%        \let\@auxout\@auxoutown
%        \nocite{##1}}}
%    \end{verbatim}
%    \begin{macrocode}
    \expandafter\edef\csname nocite\@suffix\endcsname##1{{%
      \let\noexpand\@auxout\csname @auxout\@suffix\endcsname
      \noexpand\nocite{##1}}}% 
%    \end{macrocode}
% \end{macro}
%
%
% \begin{macro}{\bibitem<s>}
%
%    \changes{v1.3d}{2005/10/25}{Define a bibliography specific
%                    version of \macrott{@bibitem} and
%                    \macrott{@biblabel} if \option{labeled} is
%                    active.}  
%
%    \begin{macrocode}
    \iflabeled % if option labeled
      \expandafter\edef\csname @bibitem\@suffix\endcsname##1{%
        \noexpand\item
        \noexpand\if@filesw \noexpand\immediate\noexpand\write\noexpand\@auxout
        {\noexpand\string\noexpand\bibcite{##1}%
           {\@suffix\noexpand\the\noexpand\value{\noexpand\@listctr}}}%
        \noexpand\fi
        \noexpand\ignorespaces}%
      \expandafter\edef\csname @biblabel\@suffix\endcsname##1{[\@suffix##1]}%
    \fi % end if option labeled
%    \end{macrocode}
% \end{macro}
%
%
% \begin{macro}{\bibliography<s>}
%    \changes{v1.0i}{2000/01/12}{Remove setting of \macrott{usecounter} and
%                                \macrott{newusecounter} which is done
%                                in redefined \macrott{bibliography}.}   
%    \changes{v1.1e}{2001/01/04}{Definition of \macrott{bibname} analogously
%                                to \macrott{refname}.}
%    \changes{v1.3d}{2005/10/25}{Activate the bibliography specific
%                    versions of \macrott{@bibitem} and
%                    \macrott{@biblabel} if \option{labeled} is
%                    active.}  
%
%    Define new |\bibliography|\meta{s} equivalents.
%    The standard |\bibliography| macro does
%    two things: The bibdata file is written to the auxiliary file
%    |\@auxout| and the |.bbl| file |\jobname.bbl| is inputed. For
%    |\bibliography|\meta{s}, we thus let both |\@auxout| 
%    and |\jobname| refer to the new auxiliary file \meta{s}|.aux|.
%    Since some styles and classes write the heading
%    of the bibliography to the table of contents, the old meaning of
%    |\@auxout| is saved in |\temp@auxout| and |\addtocontents| is locally
%    replaced by \package{multibib}s variant |\mb@addtocontents|
%    (defined above) which writes to |\temp@auxout| instead of
%    |\auxout|. Further, the heading of the bibliography generated by
%    |\refname| or |\bibname| depending on the class is set to the
%    particular heading of |\bibliography|\meta{s}. After this
%    redefinitions, the standard |\bibliography| can be evoked. 
%    Extra braces are needed to encapsulate the various |\let|s.
%    \begin{macrocode}  
    \expandafter\edef\csname bibliography\@suffix\endcsname##1{{%
      \let\noexpand\temp@auxout\noexpand\@auxout 
      \let\noexpand\addtocontents\noexpand\mb@addtocontents
      \let\noexpand\@auxout\csname @auxout\@suffix\endcsname
      \let\noexpand\jobname
         \expandafter\noexpand\csname @auxout\@suffix name\endcsname
      \let\noexpand\refname
          \expandafter\noexpand\csname refname\@suffix\endcsname
      \let\noexpand\bibname
          \expandafter\noexpand\csname refname\@suffix\endcsname
      \iflabeled
        \let\noexpand\@bibitem
            \expandafter\noexpand\csname @bibitem\@suffix\endcsname
        \let\noexpand\@biblabel
            \expandafter\noexpand\csname @biblabel\@suffix\endcsname
      \fi
      \noexpand\bibliography{##1}%
      }}
%    \end{macrocode}
% \end{macro}
%
% \newpage
%
% \begin{macro}{\bibliographystyle<s>}
% \changes{v1.3c}{2004/02/10}{Write directly the \macrott{bibstyle}
%                macro to the specific auxfile instead of redefining a
%                auxhandle and then using the standard
%                \macrott{bibliographystyle} macro.} 
%
%    Define new |\bibliographystyle|\meta{s} equivalents.
%    \begin{macrocode}  
    \expandafter\edef\csname bibliographystyle\@suffix\endcsname##1{%
      \noexpand\if@filesw
        \noexpand\immediate\noexpand\write\csname @auxout\@suffix\endcsname
          {\noexpand\string\noexpand\bibstyle{##1}}%
      \noexpand\fi}
%    \end{macrocode}
% \end{macro}
%
%
%    Close loop and end macro.
%    \begin{macrocode}
   }%
}
%    \end{macrocode}
%
%
%    Finally, restrict the use of |\newcites| to the preamble such that
%    the auxiliary files can be |\@input@|ed.
%    \begin{macrocode}
\@onlypreamble\newcites
%    \end{macrocode}
%
%
%    \begin{macrocode}
%</package>
%    \end{macrocode}
%
% \PrintIndex
% \PrintChanges
% \Finale
%
